\section{Future Works}
The multi-region transactions set three clear objectives: high throughput, low client-perceived latency (especially for IRTs), and as strong as possible consistency. Earning all three requires careful designs to strike a balance between the three individual objectives. Our exploration of RLS opens up a new design space for achieving such goals. We draw some lessons worth further research:

\begin{itemize}[leftmargin=*, itemsep=1pt]
\setlength{\parsep}{0pt}
\setlength{\parskip}{0pt}
\setlength{\parindent}{1em}
    \item \textbf{Multi-Layered Consistency Model.} Modern network exhibits a multi-layered structure~\cite{zhang2023redt, webb2011topology}. For instance, Cloud providers (e.g., Huawei~\cite{huawei:region}) are diligent in using CXL- and RDMA-based networks inside a data center, a dedicated network between data centers inside a region, and public networks across the regions. In this work, we have explored the consistency model for multi-region deployment by treating IRTs and CRTs differently. However, a more fine-grained design may still be desirable to tightly fit consistency guarantees into the network stack. For instance, in-network ordering technologies~\cite{bidl:sosp21, epaxos:sosp13, li20211pipe, choi2023hydra}, which leverage the properties of the network for order, are proposed for in-data center deployment. How to combine such technologies with geo-distributed transactions is still an open problem. A multi-layered consistency model may work as the glue to bridge the design of in-network transaction processing technologies with geo-distributed transaction processing technologies.
    \item \textbf{Data Locality and Partial Replication.} Even though RLS provides practical mitigation for achieving high throughput and low latency for both CRTs and IRTs, the cost of remote reads cannot be fundamentally removed. Therefore, an efficient data partition and replication policy are still critical for real-world usage: a good partition policy~\cite{abebe2020dynamast, abebe2020morphosys, curino2010schism, pavlo2012skew, zamanian2015locality} can vastly reduce the ratio of remote reads, and a cautious replication policy~\cite{taft2020cockroachdb, dast:eurosys21, schiper2010p} can balance the overhead of data synchronization and remote access. 
\end{itemize}